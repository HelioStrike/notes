\documentclass[titlepage, 12pt]{article}
\usepackage[parfill]{parskip}
\usepackage{amsfonts}
\usepackage{hyperref}
\usepackage{xcolor}
\usepackage{amsmath}

\hypersetup{
	colorlinks=true,
	linkcolor=blue,
	filecolor=magenta,
	urlcolor=blue,
}

\newcommand*{\logo}{\fbox{$\mathfrak{heliostriKe}$}}

\begin{document}
	
	\begin{titlepage}
		
		\raggedleft
		
		\vspace*{\baselineskip}
		
		{Sai Sandeep\\\url{https://github.com/heliostrike/notes}}
		
		\vspace*{0.167\textheight}
		
		\textbf{\LARGE Notes on}\\[\baselineskip]
		
		\textbf{\textcolor{teal}{\huge Kinematic Modeling in 2D}}\\[\baselineskip]
		
		{\Large \textit{Vehicle Dynamic Modeling}}
		
		\vfill
		
		{\large \logo}
		
		\vspace*{3\baselineskip}
		
	\end{titlepage}

	\newpage
	
	\section{Introduction}
	The vectors $\{b\textsubscript{1}, b\textsubscript{2}\}$ and $\{e\textsubscript{1}, e\textsubscript{2}\}$ define 2 different coordinate frames, $F\textsubscript{B}$ and $F\textsubscript{E}$. The vectors are about the same point but differ by some rotation $\theta$.
			
	We can then define the rotation matrices as:
	
	$C\textsubscript{EB} = 
	\begin{bmatrix}
		\cos{\theta} & \sin{\theta} \\
		-\sin{\theta} & \cos{\theta}
	\end{bmatrix}
	$

	$C\textsubscript{BE} = 
	\begin{bmatrix}
		\cos{\theta} & -\sin{\theta} \\
		\sin{\theta} & \cos{\theta}
	\end{bmatrix}
	$

	\subsection{Coordinate Transformation}
	
	Conversion between body and inertial coordinates is done with a translation vector and a rotation matrix.
	
	$P\textsubscript{B} = C\textsubscript{EB} (\theta) P\textsubscript{E} + O\textsubscript{EB}$

	$P\textsubscript{E} = C\textsubscript{BE} (\theta) P\textsubscript{B} + O\textsubscript{BE}$
	
	\subsection{Homogenous Coordinate Form}
	
	A 2D vector in can be converted to:
	
	$P = 
	\begin{bmatrix}
		x \\
		y
	\end{bmatrix}
	\rightarrow
	\bar{P} = 
	\begin{bmatrix}
		x \\
		y \\
		1
	\end{bmatrix}
	$
	
	Transforming using homogenous coordinates:
	
	$\bar{P\textsubscript{E}} = 
	\begin{bmatrix}
		C\textsubscript{EB}(\theta) | O\textsubscript{EB}
	\end{bmatrix}
	\bar{P\textsubscript{B}}
	$
	
	\subsection{2D Kinematic Modeling}
	
	The kinematic constraint is nonholonomic.
	\begin{enumerate}
		\item A constraint on rate of change of degrees of freedom.
		\item Vehicle velocity is always tangent to current path.
	\end{enumerate}

	$
	\frac{dy}{dx}\
	=
	\tan{\theta}
	=
	\frac{\sin{\theta}}{\cos{\theta}}
	$
	
	Non-holonomic constraint: $
	\dot{y}\cos{\theta}
	-
	\dot{x}\sin{\theta}
	= 0
	$
	
	Velocity components: $
	\dot{x}
	= v\cos{\theta} 
	$, 
	$ 
	\dot{y}
	= v\sin{\theta}
	$
	
	\section{Two - Wheeled Robot Kinematic Model}
	Say control inputs are wheel speeds, $v\textsubscript{1}, v\textsubscript{2}$.
	We'll need to define the following parameters.
	
	\begin{enumerate}
	
	\item Center: $p$
	\item Wheel to center: $l$
	\item Wheel radius: $r$
	\item Wheel rotation rates: $w1, w2$
		
	\end{enumerate}

	Average velocity can be computed as:
	$
	v = \frac{v\textsubscript{1} + v\textsubscript{2}}{2}
	=
	\frac{rw\textsubscript{1} + rw\textsubscript{2}}{2}
	$
	
	If $v\textsubscript{1}$ and $v\textsubscript{2}$ and different, the vehicle moves along a curved path.
	
	The rotation about the ICR (Instantaneous center of rotation) is given by, $\omega = 
	\frac{rw\textsubscript{1} - rw\textsubscript{2}}{2l}$
	
	\subsection{Kinematic Model of a Simple 2D Robot}
	
	\subsubsection{Continuous time model}
	
	$
	\dot{x} = 
	[
	(\frac{rw\textsubscript{1} + rw\textsubscript{2}}{2})\cos(\theta)
	]
	$
	
	$
	\dot{y} = 
	[
	(\frac{rw\textsubscript{1} + rw\textsubscript{2}}{2})\sin(\theta)
	]
	$
	
	$
	\dot{\theta} = 
	(
	\frac{rw\textsubscript{1} - rw\textsubscript{2}}{2l}
	)
	$
	
	\subsubsection{Discrete time model}
	
	$
	x\textsubscript{k+1}
	=
	x\textsubscript{k}
	+
	[
	(\frac{rw\textsubscript{1,k} + rw\textsubscript{2,k}}{2})\cos(\theta\textsubscript{k})
	]\delta{t}
	$
	
	$
	y\textsubscript{k+1}
	=
	y\textsubscript{k}
	+
	[
	(\frac{rw\textsubscript{1,k} + rw\textsubscript{2,k}}{2})\sin(\theta\textsubscript{k})
	]\delta{t}
	$

	$
	\theta\textsubscript{k+1} =
	\theta\textsubscript{k}
	+ 
	(
	\frac{rw\textsubscript{1,k} - rw\textsubscript{2,k}}{2l}
	)\delta{t}
	$

\end{document}